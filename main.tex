%%%%%%%%%%%%%%%%%%%%%%%%%%%%%%%%%%%%%%%%%%%%%%%%%%%%%%%%%%%%%%%%%%%%%%%%%%%%
% AGUtmpl.tex: this template file is for articles formatted with LaTeX2e,
% Modified March 2013
%
% This template includes commands and instructions
% given in the order necessary to produce a final output that will
% satisfy AGU requirements.
%
% PLEASE DO NOT USE YOUR OWN MACROS
% DO NOT USE \newcommand, \renewcommand, or \def.
%
% FOR FIGURES, DO NOT USE \psfrag or \subfigure.
%
%%%%%%%%%%%%%%%%%%%%%%%%%%%%%%%%%%%%%%%%%%%%%%%%%%%%%%%%%%%%%%%%%%%%%%%%%%%%
%
% All questions should be e-mailed to latex@agu.org.
%
%%%%%%%%%%%%%%%%%%%%%%%%%%%%%%%%%%%%%%%%%%%%%%%%%%%%%%%%%%%%%%%%%%%%%%%%%%%%
%
% Step 1: Set the \documentclass
%
% There are two options for article format: two column (default)
% and draft.
%
% PLEASE USE THE DRAFT OPTION TO SUBMIT YOUR PAPERS.
% The draft option produces double spaced output.
%
% Choose the journal abbreviation for the journal you are
% submitting to:

% jgrga JOURNAL OF GEOPHYSICAL RESEARCH
% gbc   GLOBAL BIOCHEMICAL CYCLES
% grl   GEOPHYSICAL RESEARCH LETTERS
% pal   PALEOCEANOGRAPHY
% ras   RADIO SCIENCE
% rog   REVIEWS OF GEOPHYSICS
% tec   TECTONICS
% wrr   WATER RESOURCES RESEARCH
% gc    GEOCHEMISTRY, GEOPHYSICS, GEOSYSTEMS
% sw    SPACE WEATHER
% ms    JAMES
% ef    EARTH'S FUTURE
%
%
%
% (If you are submitting to a journal other than jgrga,
% substitute the initials of the journal for "jgrga" below.)

\documentclass[ms, draft]{agutex}
% To create numbered lines:

% If you don't already have lineno.sty, you can download it from
% http://www.ctan.org/tex-archive/macros/latex/contrib/ednotes/
% (or search the internet for lineno.sty ctan), available at TeX Archive Network (CTAN).
% Take care that you always use the latest version.

% To activate the commands, uncomment \usepackage{lineno}
% and \linenumbers*[1]command, below:

\usepackage{lineno}
\linenumbers*[1]

%  To add line numbers to lines with equations:
%  \begin{linenomath*}
%  \begin{equation}
%  \end{equation}
%  \end{linenomath*}
%%%%%%%%%%%%%%%%%%%%%%%%%%%%%%%%%%%%%%%%%%%%%%%%%%%%%%%%%%%%%%%%%%%%%%%%%
% Figures and Tables
%
%
% DO NOT USE \psfrag or \subfigure commands.
%
%  Figures and tables should be placed AT THE END OF THE ARTICLE,
%  after the references.
%
%  Uncomment the following command to include .eps files
%  \usepackage[dvips]{graphicx}
%
%  Uncomment the following command to allow illustrations to print
%   when using Draft:
%  \setkeys{Gin}{draft=false}
%
% Substitute one of the following for [dvips] above
% if you are using a different driver program and want to
% proof your illustrations on your machine:
%
% [xdvi], [dvipdf], [dvipsone], [dviwindo], [emtex], [dviwin],
% [pctexps],  [pctexwin],  [pctexhp],  [pctex32], [truetex], [tcidvi],
% [oztex], [textures]
%
% See how to enter figures and tables at the end of the article, after
% references.
%
%% ------------------------------------------------------------------------ %%
%
%  ENTER PREAMBLE
%
%% ------------------------------------------------------------------------ %%

% Author names in capital letters:
\authorrunninghead{HAMMAN ET AL.}

% Shorter version of title entered in capital letters:
\titlerunninghead{LAND-OCEAN COUPLING IN RASM}

%Corresponding author mailing address and e-mail address:
\authoraddr{Corresponding author: Bart Nijssen,
Department of Civil \& Environmental Engineering, Box 352700,
University of Washington, Seattle, WA 98195-2700, USA.
(nijssen@uw.edu)}

\begin{document}

%% ------------------------------------------------------------------------ %%
%
%  TITLE
%
%% ------------------------------------------------------------------------ %%


\title{Land-Ocean Coupling in the Regional Arctic System Model}

%% ------------------------------------------------------------------------ %%
%
%  AUTHORS AND AFFILIATIONS
%
%% ------------------------------------------------------------------------ %%


% Use \author{\altaffilmark{}} and \altaffiltext{}

% \altaffilmark will produce footnote;
% matching \altaffiltext will appear at bottom of page.

\authors{Joseph Hamman,\altaffilmark{1}
Bart Nijssen,\altaffilmark{1},
Anthony Craig,\altaffilmark{2}
Wieslaw Maslowski\altaffilmark{2},
Robert Osinski\altaffilmark{3},
and Andrew Roberts\altaffilmark{2}}

\altaffiltext{1}{Department of Civil \& Environmental Engineering,
University of Washington, Seattle, WA, USA.}
\altaffiltext{2}{Department of Oceanography, Naval Postgraduate School,
Monterey, CA, USA.}
\altaffiltext{3}{Polish Institute of Oceanology, Sopot, Poland.}


%% ------------------------------------------------------------------------ %%
%
%  ABSTRACT
%
%% ------------------------------------------------------------------------ %%

% >> Do NOT include any \begin...\end commands within
% >> the body of the abstract.

\begin{abstract}
The purpose of the abstract is twofold: (1) state the nature of the investigation and (2) summarize the important conclusions of this investigation. The abstract should be suitable for separate publication in an abstract journal and be adequate for indexing. Make sure to check for the following:

It is set as a single paragraph.
It is limited to 250 words for all journals except GRL where the limit is 150 words.
It does not include table or figure mentions.
If it has reference citations that are part of the sentence, they should be in roman type and have parentheses around the year; parenthetical reference citations will be deleted.
All abbreviations used in the abstract are defined.
\end{abstract}

%% ------------------------------------------------------------------------ %%
%
%  BEGIN ARTICLE
%
%% ------------------------------------------------------------------------ %%

% The body of the article must start with a \begin{article} command
%
% \end{article} must follow the references section, before the figures
%  and tables.

\begin{article}

%% ------------------------------------------------------------------------ %%
%
%  TEXT
%
%% ------------------------------------------------------------------------ %%

\section{Introduction}

The freshwater flux from the land to the ocean, via runoff, is the largest single contributor of freshwater to the Arctic Ocean.
The seasonal freshwater flux from the land to the ocean plays an important role in coastal ocean hydrography and dynamics as well as to sea ice formation and melt.
Runoff from Arctic river basins is the primary source of buoyancy-driven coastal currents, such as the Alaska Coastal Current (ACC), Siberian Coastal Current, Norwegian Coastal Current, and East Greenland Coastal Current \citep[e.g.][]{Morison_2000,Boyd_2002,McGeehan_2012}.
Such currents redistribute both fresh water and heat, which locally play important roles in the shelf dynamics and shelf-basin interaction. Another important aspect is the effect of buoyancy and heat fluxes delivered by rivers on the onset of sea ice formation in winter and melt in spring/summer.
First, less salty water freezes at higher temperature, i.e. it does not have to be cooled as much as higher salinity water to freeze.
Thus, for a warming and freshening arctic, the onset of freezing in areas highly influenced by streamflow may be partially buffered against regional warming.
Second, heat delivered with runoff combined with the ice-albedo effect can make a big difference on the local retreat of sea ice from a shelf.
Finally, long-term river runoff maintains the surface freshwater layer in the Arctic Ocean and is a critical and necessary source required to balance freshwater export through Fram Strait and the Canadian Arctic Archipelago into the North Atlantic.

Approximately 11\% of the global terestrial runoff drains into the Arctic Ocean, which holds only 1\% of the Earth's saltwater \citep{Lewis_2000,Lammers_2001}.
Streamflow comprises approximately 38\% of the total freshwater input into Arctic Ocean \cite{Serreze_2006}.
Across the Arctic region, the annual runoff hydrograph is characterized by a large spring freshet, with approximately 60\% of the annual runoff volume occurring between April and July \citep{Lammers_2001}.

Numerous observational studies have explored the seasonal and inter-annual behavior of Arctic runoff.
\citet{Lammers_2001} compiled R-ArcticNET, a regional hydrographic database of monthly mean streamflow records, including over 3700 streamflow gauges.
This dataset was used by \citet{Shiklomanov_2009} in their investigation of increasing river discharge in the largest Eurasian rivers and by \citep{Tan_2011} in their investigation of changes in spring snowmelt timing.

% [uncoupled modeling studies]
Adam 2007, 2008, Tan 2011, Su 2007?
\citep{Dai_2009}
\citep{Dai_2002}
Note: this could use some non UW references

% [observational studies of arctic runoff from the ocean perspective]
\citep{Rabe_2011}
\citep{Fichot_2013}
\citep{McPhee_1998}
\citep{Macdonald_2002}
\citep{Timmermans_2011}
\citep{Nummelin_2015}

% [ocean / sea ice specific studies]
The importance of the terestrial freshwater flux has been shown to be an important driver of ice-ocean dynamics in coupled General Circulation Models (GCMs).
\citet{Newman_2008} applied observed climatological runoff at the largest nine rivers in the Arctic basin and used dye tracers to visualize the spatial distribution of runoff.
They found the highest concetrations of river runoff along the Siberian coast.

% [other climate models that include coupled land-ocean dynamics]
Previous studies have coupled streamflow routing models to within coupled climate models.
Cell-to-cell routing methods, such as the River Transport Model \citep[RTM][]{Branstetter_2003} have been applied globally in a number of GCMs (e.g. CESM) and are typically difficult to parameterize over a range of spatial scales \citep{Sushama_2004. 
\citep{Olivera_2000} introduced a source-to-sink streamflow routing approach, in GCMs. %need to expand this section

In fact, \citet{Bring_2015}, in their synthesis of CMIP5 runoff dynamics, conclude that a consented community effort should be made to improve the understanding and modeling of basin scale freshwater fluxes in coupled climate modeling.

Here we describe the land-ocean coupling in the Regional Arctic System Model (RASM).
RASM is a high resolution fully coupled regional Earth system model (cite) applied over the pan-Arctic domain (Fig 1).
Development of RASM aims to improve the representation and understanding of complex coupled processes in the Arctic climate system.
RASM uses the RVIC streamflow routing model to route runoff from the Variable Infiltration Capacity \citep[VIC][]{Liang_1996} land surface model to the Parallel Ocean Program (POP; cite) ocean model, utilizing the Community Earth System Model (CESM; cite) flux coupling infrastructure (cite).
For the purposes of this paper, we are principally concerned with the coastal streamflow flux and its role in the Arctic Ocean system.

\section{Models}

\subsection{RASM}

The Regional Arctic System Model (RASM) is a fully coupled, high spatial and high temporal resolution regional climate model that has been recently developed under the support of the Department of Energy Earth System Modeling program.
Principle goals of the modeling project are to better understand the interaction between physical systems in the Arctic drainage basin, to advance understanding of past and present states of Arctic climate, and to improve seasonal to multi-decadal prediction capabilities of key Arctic indicators.
Existing model components, typically run off-line of one another are coupled using the Community Earth System Model (CESM) coupled model framework and the CPL7 flux coupler \cite{Craig_2011}.
Below, we provide a brief description of the five component models in RASM version 1.0.
The reader is encouraged to see the cited RASM specific references for more information on the specific implementation of individual component models.

\begin{enumerate}
\item CICE: The Los Alamos Sea Ice model (Hunke et al. 2013) is physically based dynamic sea ice model.
Roberts et al. [Manuscript in Preparation] provide a description of the application of CICE, version 5, within RASM.
\item POP: The Parallel Ocean Program model \cite{smith_2010} is a general circulation ocean model.
Osinski et al. [Manuscript in Preparation] provide a description of the application of POP, version 2, within RASM.
Relevant to this study, POP uses a virtual salinity flux to represent changes in ocean salinty due to surface fluxes of freshwater (runoff, precipitation, and evaporation).
\item VIC: The Variable Infiltration Capacity model \cite{Liang_1996} is a macroscale land surface hydrology model.
Hamman et al. [In Review] provide a description of the application of VIC, version 4, within RASM.
\item WRF: The Weather Research and Forecasting atmospheric model \citep{Skamarock_2007} is a mesoscale meteorological model.
DuVivier and Cassano [Manuscript in Preparation] and Cassano et al. [Manuscript in Preparation] provide a detailed description of the WRF model, version 3.2, as it is applied in RASM.
\item RVIC: The RVIC streamflow routing model is an adapted version of the \citet{Lohmann_1996} linear routing model frequently used to route streamflows from VIC output.
A detailed description of the RVIC model is provided in section 2.2.
\end{enumerate}

In RASM Version 1.0, the land, atmosphere, and runoff components share a 50-km near-equal-area North Pole stereographic grid mesh.
The ocean and sea ice models share a 1/12° rotated stereographic grid mesh.
Models are coupled every 20 minutes in a configuration described by \citet{Roberts_2015}. 

\subsection{RVIC}

The RVIC streamflow routing model is a modified version of the routing model typically used to post-process VIC model output \cite{Lohmann_1996, Lohmann_1998a}.
The original \citet{Lohmann_1996} model has been used in many offline modeling studies across spatial scales from 1/16$^{\circ}$ to 2$^{\circ}$ \citep[.e.g.][]{Nijssen_1997,Lohmann_1998b,Su_2005}.
The routing model is a source-to-sink model that solves a linearized version of the Saint-Venant equations.
The model has two distinct steps, the preprocessing step where linear time invariant impulse response functions (IRFs) are developed (see sections 2.2.1 and 2.2.2), and a the convolution step where distributed runoff from the land surface model in routed to downstream points (2.2.3).

The RVIC model differs from the original \citet{Lohmann_1996} model in four ways:

\begin{enumerate}
\item RVIC completely separates the development of the IRFs from the flow convolution,
\item RVIC allows the development of the Impulse Response Functions to be done on flow direction grids that do not match that of the land model (see section 2.1.2),
\item The RVIC convolution scheme operates in a space-before-time pattern, facilitating direct coupling with land surface models (see section 2.1.3),
\item RVIC includes many infrastructure improvements, including the ability to store the model state and to read and write netCDF files.
\end{enumerate}

\subsubsection{Impulse Response Function Development}

The source-to-sink response of flow from any model grid cell to a downstream point is represented as a linear and time invariant response to an impulse of runoff.
Therefore, the development of the IRF between any source and sink points is only a function of the horizontal travel time of water within the source grid cell and to the downstream point with the addition of a flow diffusion parameter.
The horizontal travel distance is computed using a flow direction raster \citep[e.g.][]{Wu_2011}
The Saint-Venant equation

 \begin{equation}
     \frac{\partial Q}{\partial t} = D \frac{\partial^2 Q}{\partial x} + C \frac{\partial Q}{\partial x}
 \end{equation}
 
represents the flow $Q$, at time $t$, at a downstream point $x$, as a function of the wave velocity, $C$, and the diffusivity, $D$; both of which may be estimated from geographical data.

Eq. (1) can be solved with convolution integrals

 \begin{equation}
	Q(x,t) = \int_0^t U(t-s)h(x,s)ds
 \end{equation}

where

 \begin{equation}
	h(x, t) = \frac{x}{2t\sqrt{\pi tD}}exp\left(-\frac{(Ct-x)^2}{4Dt}\right)
 \end{equation}

is Green’s impulse response function, and $U$ is a unit hydrograph representing the distribution of flow from within each source grid cell to the edge of of the source grid cell.
The Saint-Venant equations are solved for the flow response from every source to outlet pair.

\subsubsection{Upscaling and Basin Aggregation}

The development of the impulse response functions may be done on a high resolution (e.g. 1/16$^{\circ}$) flow network such as the one developed by \citet{Wu_2011}.
Whereas the typical employment of the \citet{Lohmann_1996} model requires that the flow direction grid and the land surface model grid match, the RVIC implementation first upscales and aggregates the high resolution IRF grid to the RASM grid (Figure 2).
The upscaling process uses the first-order conservative remapping technique developed by \cite{Jones_1999}.
Because the remapping scheme is conservative, the unit response to runoff (unit hydrograph) from each source grid cell is maintained.
Finally, the regridded IRFs are aggregated to include all basins flowing into a single outlet grid cell.

\subsubsection{Convolution}

With the development of the IRFs complete, RVIC performs a simple linear convolution of the IRFs and runoff fluxes from the land model.
In essence, RVIC aggregates the flow contribution from all upstream grid cells at every timestep, distributing them in time according the IRF.

\subsubsection{RVIC in RASM}

The IRFs used in RASM were developed using the \citet{Wu_2011} 1/16$^{\circ}$ flow direction rasters.
We used a spatially constant flow velocity and diffusion of 1 m/s and a 2,000 m$^2$/s, respectively.
IRFs were developed for every coastal 1/16$^{\circ}$ grid cell and were aggregated and remapped onto the 50-km near equal area land surface grid.

\subsection{Coupling}

In nature, turbulent mixing and other diffusive processes combine to gradually spread freshwater along the coast and into the open ocean.
In the modeling environment however, these processes are difficult to represent at the spatial resolution that ocean and sea ice models are run at.
To simulate the dispersion of freshwater throughout each ocean grid cell within RASM, a diffusion scheme is applied to avoid unrealistic salinity gradients that would occur when a river’s entire outflow is applied to one ocean grid cell.
The mapping from the runoff grid to the ocean grid is produced as a preprocessing step using the runoff and ocean grids and masks.
Each runoff gridcell is mapped to the nearest ocean grid cell.
The flux is then smoothed over all gridpoints in a 300km radius with distance weighting efolding scale of 1000 km such that the total quantity is conserved from runoff to ocean.
This smoothing prevents excessively large and unrealistic runoff fluxes into any single ocean grid cell and encourages computational stability within the ocean model while representing the distributed changes in salinity due to the land-ocean freshwater flux.

\section{Model Simulations and Data}

In this paper, we will present results from two RASM simulations, one including the runoff flux to the ocean and the other without it, hereafter referred to as $RASM_A$ and $RASM_B$.
Both simulations use lateral boundary conditions derived from the ERA-Interim Reanalysis \citep{Dee_2011} and were run from September 1, 1979 through December 31, 2014.
Model components were coupled every 20 minutes.
  
We compare output from RASM simulations to a streamflow observations from the R-ArcticNET database \citep{Lammers_2001}.
This dataset provides mean monthly streamflow observations at XX locations within the RASM domain and analysis period.

\citep{Dai_2009}

\section{Results}

\subsection{Coupled streamflow routing results}

We begin by extending the analysis of Hamman et al. [Manuscript in Preparation] to assess the annual streamflow hydrograph.
Following Hamman et al. [Manuscript in Preparation], we compare RASM output, routed offline using RVIC to the R-ArcticNET database.
Figure X shows the annual hydrograph at six select streamflow gauge locations.
At each of these streamflow gauge locations, RASM simulates the spring freshet in May or June, although typically peaking the month before the observations.

Figure Xa compares the fraction of the annual flow volume that occurs between April and July between the RASM$_ERA$ simulation and the observations.
Figure Xb compares the center of mass of the annual hydrograph between the RASM$_ERA$ simulation and the observations.

TODO:
Make scatter plots of April-July fractional volume, centroid of timing (RASM vs obs).
Make table showing same data for selection of main rivers.

\subsection{Role of high resolution coastal streamflow on ocean and sea ice dynamics}

We begin by exploring the influence of coastal streamflow on the surface salinity and temperature in individual ocean basins.
We compare two fully coupled simulations (RASM$_ERA$ and RASM$_CFSR$) with one partially coupled simulation (RASM$_ERA_NOROF$).
Descriptions of these three simulations are included in section 3.
In each of the four ocean basins shown in Figure X, the simulations including the runoff flux to the ocean lower the salinity in the surface layer of the ocean between 1-3 g/kg.
Accompanying the lower salinty, is a shift in the annual temperature cycle in the surface layer with a general reduction in temperature in the summer months.

This section to be completed after analysis of ocean/sea ice model results.
\begin{itemize}
\item Comparison between RB case with surface restoring to fully coupled RBR case.
\item Coastal ocean circulation in RB / RBR cases
\item Sea ice production as a function of salinity changes
\item Sea ice thickness distribution over decadal timescales, connections to freshwater fluxes.
\end{itemize}

\section{Discussion}

\subsection{Large Scale Arctic Freshwater Budget}
The addition of RVIC within RASM has allowed us to close the large scale freshwater budget in central Arctic.
Figure X presents the major fluxes and storages between the land, ocean, and atmosphere components in RASM, and compares them to the values presented in \citet{Serreze_2006}.
As also shown by Hamman et al. [Manuscript in Preparation], the biases in precipitation over land and runoff to the central Arctic are relatively low, compared to observations and reanalysis products.
The precipitation and evaporation fluxes over the ocean are much less constrained than those over land.
In general, RASM tends to have less precipitation and evporation over the cetral Arctic ocean.
Place holder for discussion about ocean fluxes.

\subsection{Routing Processes}
The RVIC model is a source-to-sink linear routing model that uses time-invarient IRFs to route streamflow to coastal grid cells in RASM.
At present, there is no mechanism in the RVIC model to account for non-linear routing processes such as overbank flow, wetlands, ice jams, reservoir operations, and industrial or aggricultural diversions.
\citet{Adam_2007} highlight the importance of representing the reservoir influences in order to capture the annual hydrograph in some Eurasian rivers.
Errors caused by not explicitly representing these processes are aparrent in figure X.  For example, in the Nelson River, which is highly influenced by reservoir operations, RASM produces a naturalized hydrograph that does not resemble the observed hydrograph.
Comment: ice dams are also not resolved in RASM although the results show here are a timescale that smooth that.


B.	Need for other runoff properties such as temperature/heat flux, nutrients, and sediments.
Vliet1 et al 2012: Coupled daily streamflow and water temperature modelling in large river basins

C.	Need for thermodynamic scheme that resolves changes in freezing temperature as a function of salinity

\section{Conclusions}


%%% End of body of article:

%%%%%%%%%%%%%%%%%%%%%%%%%%%%%%%%
%% Optional Appendix goes here
%
% \appendix resets counters and redefines section heads
% but doesn't print anything.
% After typing \appendix
%
%\section{Here Is Appendix Title}
% will show
% Appendix A: Here Is Appendix Title
%
%%%%%%%%%%%%%%%%%%%%%%%%%%%%%%%%%%%%%%%%%%%%%%%%%%%%%%%%%%%%%%%%
%
% Optional Glossary or Notation section, goes here
%
%%%%%%%%%%%%%%
% Glossary is only allowed in Reviews of Geophysics
% \section*{Glossary}
% \paragraph{Term}
% Term Definition here
%
%%%%%%%%%%%%%%
% Notation -- End each entry with a period.
% \begin{notation}
% Term & definition.\\
% Second term & second definition.\\
% \end{notation}
%%%%%%%%%%%%%%%%%%%%%%%%%%%%%%%%%%%%%%%%%%%%%%%%%%%%%%%%%%%%%%%%
%
%  ACKNOWLEDGMENTS

\begin{acknowledgments}
This research was supported under United States Department of Energy (DOE) grants DE-FG02-07ER64460 and DE-SC0006856 to the University of Washington, and XXXXXXXXXX to the Naval Post Graduate School.
Supercomputing resources were provided through the United States Department of Defense (DOD) High Performance Computing Modernization Program at the Army Engineer Research and Development Center and the Air Force Research Laboratory.
\end{acknowledgments}

%% ------------------------------------------------------------------------ %%
%%  REFERENCE LIST AND TEXT CITATIONS

\bibliographystyle{agu08}
\bibliography{biblio}

% Either type in your references using
% \begin{thebibliography}{}
% \bibitem{}
% Text
% \end{thebibliography}
%
% Or,
%
% If you use BiBTeX for your references, please use the agufull08.bst file (available at % ftp://ftp.agu.org/journals/latex/journals/Manuscript-Preparation/) to produce your .bbl
% file and copy the contents into your paper here.
%
% Follow these steps:
% 1. Run LaTeX on your LaTeX file.
%
% 2. Make sure the bibliography style appears as \bibliographystyle{agufull08}. Run BiBTeX on your LaTeX
% file.
%
% 3. Open the new .bbl file containing the reference list and
%   copy all the contents into your LaTeX file here.
%
% 4. Comment out the old \bibliographystyle and \bibliography commands.
%
% 5. Run LaTeX on your new file before submitting.
%
% AGU does not want a .bib or a .bbl file. Please copy in the contents of your .bbl file here.

%\begin{thebibliography}{}

%\providecommand{\natexlab}[1]{#1}
%\expandafter\ifx\csname urlstyle\endcsname\relax
%  \providecommand{\doi}[1]{doi:\discretionary{}{}{}#1}\else
%  \providecommand{\doi}{doi:\discretionary{}{}{}\begingroup
%  \urlstyle{rm}\Url}\fi
%
%\bibitem[{\textit{Atkinson and Sloan}(1991)}]{AtkinsonSloan}
%Atkinson, K., and I.~Sloan (1991), The numerical solution of first-kind
%  logarithmic-kernel integral equations on smooth open arcs, \textit{Math.
%  Comp.}, \textit{56}(193), 119--139.
%
%\bibitem[{\textit{Colton and Kress}(1983)}]{ColtonKress1}
%Colton, D., and R.~Kress (1983), \textit{Integral Equation Methods in
%  Scattering Theory}, John Wiley, New York.
%
%\bibitem[{\textit{Hsiao et~al.}(1991)\textit{Hsiao, Stephan, and
%  Wendland}}]{StephanHsiao}
%Hsiao, G.~C., E.~P. Stephan, and W.~L. Wendland (1991), On the {D}irichlet
%  problem in elasticity for a domain exterior to an arc, \textit{J. Comput.
%  Appl. Math.}, \textit{34}(1), 1--19.
%
%\bibitem[{\textit{Lu and Ando}(2012)}]{LuAndo}
%Lu, P., and M.~Ando (2012), Difference of scattering geometrical optics
%  components and line integrals of currents in modified edge representation,
%  \textit{Radio Sci.}, \textit{47},  RS3007, \doi{10.1029/2011RS004899}.

%\end{thebibliography}

%Reference citation examples:

%...as shown by \textit{Kilby} [2008].
%...as shown by {\textit  {Lewin}} [1976], {\textit  {Carson}} [1986], {\textit  {Bartholdy and Billi}} [2002], and {\textit  {Rinaldi}} [2003].
%...has been shown [\textit{Kilby et al.}, 2008].
%...has been shown [{\textit  {Lewin}}, 1976; {\textit  {Carson}}, 1986; {\textit  {Bartholdy and Billi}}, 2002; {\textit  {Rinaldi}}, 2003].
%...has been shown [e.g., {\textit  {Lewin}}, 1976; {\textit  {Carson}}, 1986; {\textit  {Bartholdy and Billi}}, 2002; {\textit  {Rinaldi}}, 2003].

% ...as shown by \citet{jskilby}.
%...as shown by \citet{lewin76}, \citet{carson86}, \citet{bartoldy02}, and \citet{rinaldi03}.
%...has been shown \citep{jskilbye}.
%...has been shown \citep{lewin76,carson86,bartoldy02,rinaldi03}.
%...has been shown \citep [e.g.,][]{lewin76,carson86,bartoldy02,rinaldi03}.
%
% Please use ONLY \citet and \citep for reference citations.
% DO NOT use other cite commands (e.g., \cite, \citeyear, \nocite, \citealp, etc.).

%% ------------------------------------------------------------------------ %%
%
%  END ARTICLE
%
%% ------------------------------------------------------------------------ %%
\end{article}
%
%
%% Enter Figures and Tables here:
%
% DO NOT USE \psfrag or \subfigure commands.
%
% Figure captions go below the figure.
% Table titles go above tables; all other caption information
%  should be placed in footnotes below the table.
%
%----------------
% EXAMPLE FIGURE
%
% \begin{figure}
% \noindent\includegraphics[width=20pc]{samplefigure.eps}
% \caption{Caption text here}
% \label{figure_label}
% \end{figure}


%
% ---------------
% EXAMPLE TABLE
%
%\begin{table}
%\caption{Time of the Transition Between Phase 1 and Phase 2\tablenotemark{a}}
%\centering
%\begin{tabular}{l c}
%\hline
% Run  & Time (min)  \\
%\hline
%  $l1$  & 260   \\
%  $l2$  & 300   \\
%  $l3$  & 340   \\
%  $h1$  & 270   \\
%  $h2$  & 250   \\
%  $h3$  & 380   \\
%  $r1$  & 370   \\
%  $r2$  & 390   \\
%\hline
%\end{tabular}
%\tablenotetext{a}{Footnote text here.}
%\end{table}

% See below for how to make sideways figures or tables.

\end{document}

%%%%%%%%%%%%%%%%%%%%%%%%%%%%%%%%%%%%%%%%%%%%%%%%%%%%%%%%%%%%%%%

More Information and Advice:

%% ------------------------------------------------------------------------ %%
%
%  SECTION HEADS
%
%% ------------------------------------------------------------------------ %%

% Capitalize the first letter of each word (except for
% prepositions, conjunctions, and articles that are
% three or fewer letters).

% AGU follows standard outline style; therefore, there cannot be a section 1 without
% a section 2, or a section 2.3.1 without a section 2.3.2.
% Please make sure your section numbers are balanced.
% ---------------
% Level 1 head
%
% Use the \section{} command to identify level 1 heads;
% type the appropriate head wording between the curly
% brackets, as shown below.
%
%An example:
%\section{Level 1 Head: Introduction}
%
% ---------------
% Level 2 head
%
% Use the \subsection{} command to identify level 2 heads.
%An example:
%\subsection{Level 2 Head}
%
% ---------------
% Level 3 head
%
% Use the \subsubsection{} command to identify level 3 heads
%An example:
%\subsubsection{Level 3 Head}
%
%---------------
% Level 4 head
%
% Use the \subsubsubsection{} command to identify level 3 heads
% An example:
%\subsubsubsection{Level 4 Head} An example.
%
%% ------------------------------------------------------------------------ %%
%
%  IN-TEXT LISTS
%
%% ------------------------------------------------------------------------ %%
%
% Do not use bulleted lists; enumerated lists are okay.
% \begin{enumerate}
% \item
% \item
% \item
% \end{enumerate}
%
%% ------------------------------------------------------------------------ %%
%
%  EQUATIONS
%
%% ------------------------------------------------------------------------ %%

% Single-line equations are centered.
% Equation arrays will appear left-aligned.

Math coded inside display math mode \[ ...\]
 will not be numbered, e.g.,:
 \[ x^2=y^2 + z^2\]

 Math coded inside \begin{equation} and \end{equation} will
 be automatically numbered, e.g.,:
 \begin{equation}
 x^2=y^2 + z^2
 \end{equation}

% IF YOU HAVE MULTI-LINE EQUATIONS, PLEASE
% BREAK THE EQUATIONS INTO TWO OR MORE LINES
% OF SINGLE COLUMN WIDTH (20 pc, 8.3 cm)
% using double backslashes (\\).

% To create multiline equations, use the
% \begin{eqnarray} and \end{eqnarray} environment
% as demonstrated below.
\begin{eqnarray}
  x_{1} & = & (x - x_{0}) \cos \Theta \nonumber \\
        && + (y - y_{0}) \sin \Theta  \nonumber \\
  y_{1} & = & -(x - x_{0}) \sin \Theta \nonumber \\
        && + (y - y_{0}) \cos \Theta.
\end{eqnarray}

%If you don't want an equation number, use the star form:
%\begin{eqnarray*}...\end{eqnarray*}

% Break each line at a sign of operation
% (+, -, etc.) if possible, with the sign of operation
% on the new line.

% Indent second and subsequent lines to align with
% the first character following the equal sign on the
% first line.

% Use an \hspace{} command to insert horizontal space
% into your equation if necessary. Place an appropriate
% unit of measure between the curly braces, e.g.
% \hspace{1in}; you may have to experiment to achieve
% the correct amount of space.


%% ------------------------------------------------------------------------ %%
%
%  EQUATION NUMBERING: COUNTER
%
%% ------------------------------------------------------------------------ %%

% You may change equation numbering by resetting
% the equation counter or by explicitly numbering
% an equation.

% To explicitly number an equation, type \eqnum{}
% (with the desired number between the brackets)
% after the \begin{equation} or \begin{eqnarray}
% command.  The \eqnum{} command will affect only
% the equation it appears with; LaTeX will number
% any equations appearing later in the manuscript
% according to the equation counter.
%

% If you have a multiline equation that needs only
% one equation number, use a \nonumber command in
% front of the double backslashes (\\) as shown in
% the multiline equation above.

%% ------------------------------------------------------------------------ %%
%
%  SIDEWAYS FIGURE AND TABLE EXAMPLES
%
%% ------------------------------------------------------------------------ %%
%
% For tables and figures, add \usepackage{rotating} to the paper and add the rotating.sty file to the folder.
% AGU prefers the use of {sidewaystable} over {landscapetable} as it causes fewer problems.
%
% \begin{sidewaysfigure}
% \includegraphics[width=20pc]{samplefigure.eps}
% \caption{caption here}
% \label{label_here}
% \end{sidewaysfigure}
%
%
%
% \begin{sidewaystable}
% \caption{}
% \begin{tabular}
% Table layout here.
% \end{tabular}
% \end{sidewaystable}
%
%